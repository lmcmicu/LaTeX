\documentclass{beamer}
% use the handout option to collapse delay elements (uncover, only, etc).
%\documentclass[handout]{beamer}

% remove navigation markers
\setbeamertemplate{navigation symbols}{}

% show not yet uncovered items slightly
%\beamertemplatetransparentcoveredmedium

% include frame numbers at the bottom of the frame
\setbeamertemplate{footline}[frame number]

% good plain jane themes:
\usetheme{default}
%\usetheme{Boadilla}

% good themes to show circles on the top bar
%\usetheme{Darmstadt}
%\usetheme{Frankfurt} % no subsections shown (more space)

% good theme for side bar
%\usetheme{Marburg}

% Allows change emphasis to underline, plus other useful functions. Currently set to normal emphasis.
\usepackage[normalem]{ulem}

% for nicer columns than the default
\usepackage{multicol}

% for bibliography:
\usepackage{natbib}

% Customised note page:
\setbeamertemplate{note page}{%
  {\scriptsize \mbox{}\\[6pt] \hfill Notes for slide \insertframenumber \\[2pt] \hrule}
  {\small \insertnote}
}
% Other ready-made note page templates:
% \setbeamertemplate{note page}[default]
% \setbeamertemplate{note page}[compress]
% \setbeamertemplate{note page}[lined]
% For more options see section 19 of the beamer user guide:
% http://tug.ctan.org/macros/latex/contrib/beamer/doc/beameruserguide.pdf

% Note that this must be set after including natbib otherwise you will get a compile error
% (see https://github.com/josephwright/beamer/issues/158)
% \setbeameroption{hide notes}
\setbeameroption{show notes}
% \setbeameroption{show only notes}

% font:
\usepackage{euler} % for ``blackboard'' style math fonts
\usefonttheme[onlymath]{serif}

% more or less indispensable when using math
\usepackage{amsmath}

% extra math fonts
\usepackage{amssymb}
\usepackage{amsfonts}
\usepackage{dsfont}

% essential if using the includegraphics command:
\usepackage{graphicx}

% used to put frames around images:
\usepackage{framed}

% essential:
\usepackage[english]{babel}

% for customised enumerates and itemizes
\usepackage{enumitem}

% To support working with quantum circuit diagrams, uncomment the statement below and then copy
% the file qcircuit_custom_commands.tex from the ArticleTemplates/ folder
% to the current working directory or somewhere else visible to latex.
% \usepackage{qcircuit}
 
\newcommand\xgatefull{%
  \Qcircuit @C=1.5em @R=.8em @! {
    \lstick{| 0 \rangle} & \gate{X} & \rstick{| 1 \rangle} \qw \\
    \lstick{| 1 \rangle} & \gate{X} & \rstick{| 0 \rangle} \qw}}
 
\newcommand\ygatefull{%
  \Qcircuit @C=1.5em @R=.8em @! {
    \lstick{| 0 \rangle} & \gate{Y} & \rstick{i| 1 \rangle} \qw \\
    \lstick{| 1 \rangle} & \gate{Y} & \rstick{-i| 0 \rangle} \qw}}
 
\newcommand\zgatefull{%
  \Qcircuit @C=1.5em @R=.8em @! {
    \lstick{| 0 \rangle} & \gate{Z} & \rstick{| 0 \rangle} \qw \\
    \lstick{| 1 \rangle} & \gate{Z} & \rstick{-| 1 \rangle} \qw}}
 
\newcommand\rphasegatefull{%
  \Qcircuit @C=1.5em @R=.8em @! {
    \lstick{| 0 \rangle} & \gate{R} & \rstick{| 0 \rangle} \qw \\
    \lstick{| 1 \rangle} & \gate{R} & \rstick{i| 1 \rangle} \qw}}
 
\newcommand\sphasegatefull{%
  \Qcircuit @C=1.5em @R=.8em @! {
    \lstick{| 0 \rangle} & \gate{S} & \rstick{| 0 \rangle} \qw \\
    \lstick{| 1 \rangle} & \gate{S} & \rstick{e^{i\pi/4}| 1 \rangle} \qw}}
 
\newcommand\hgatefull{%
  \Qcircuit @C=1.5em @R=.8em @! {
    \lstick{| 0 \rangle} & \gate{H} & \rstick{\frac{| 0 \rangle + | 1 \rangle}{\sqrt 2}} \qw \\
    \lstick{| 1 \rangle} & \gate{H} & \rstick{\frac{| 0 \rangle - | 1 \rangle}{\sqrt 2}} \qw}}
 
\newcommand\igate{%
  \Qcircuit @C=0.1em @R=.1em @! {
    & \gate{I} & \qw}}

\newcommand\xgate{%
  \Qcircuit @C=0.1em @R=.1em @! {
    & \gate{X} & \qw}}

\newcommand\ygate{%
  \Qcircuit @C=0.1em @R=.1em @! {
    & \gate{Y} & \qw}}

\newcommand\zgate{%
  \Qcircuit @C=0.1em @R=.1em @! {
    & \gate{Z} & \qw}}

\newcommand\rphasegate{%
  \Qcircuit @C=0.1em @R=.1em @! {
    & \gate{R} & \qw}}

\newcommand\sphasegate{%
  \Qcircuit @C=0.1em @R=.1em @! {
    & \gate{S} & \qw}}

\newcommand\hgate{%
  \Qcircuit @C=0.1em @R=.1em @! {
    & \gate{H} & \qw}}

\newcommand\cnotgate{%
  \Qcircuit @C=.1em @R=.1em @! {
    & \ctrl{1} & \qw \\
    & \targ & \qw}}

\newcommand\toffoligate{%
  \Qcircuit @C=.2em @R=.2em @! {
    & \ctrl{1} & \qw \\
    & \ctrl{1} & \qw \\
    & \targ & \qw}}

\newcommand\measgate{%
  \Qcircuit @C=1.5em @R=.8em @! {
    & \meter}}

% arXiv seems to need this for qcircuit diagrams to compile. This might not be needed any
% longer. If so, please remove this line from the template.
% \renewcommand{\Qcircuit}[1][0em]{\xymatrix @*=<#1>}


% To omit the horizontal line before footnotes:
\renewcommand{\footnoterule}{}

% useful fraction macros
\newcommand\frachalf{%
  \frac{1}{2}}

\newcommand\fracroot{%
  \textstyle{\frac{1}{\sqrt 2}}}

% for nicer fractions
\usepackage{nicefrac}

% extra highlighting colours:
\definecolor{neongreen}{RGB}{57,255,20}
\definecolor{medgrey}{RGB}{169,169,169}
\definecolor{lightgrey}{RGB}{232,232,232}
\definecolor{darkslategrey}{RGB}{47,79,79}

% background colour customisation
\setbeamercolor{background canvas}{bg=darkslategrey}

% have a blackboard image for your background instead (use the keepaspectratio
% option to not have image over the footer). Note that you must copy (or
% softlink to) the blackboard.eps file located inside the ~/texmf/tex/latex/
% directory.
%\setbeamertemplate{background}{\includegraphics[
%    width=\paperwidth,
%    height=\paperheight]{blackboard}}

% set text colors for different objects
\setbeamercolor{frametitle}{fg=white}
\setbeamercolor{structure}{fg=white}
\setbeamercolor{normal text}{fg=white}
\setbeamercolor{example text}{fg=white}
%\setbeamerfont{alerted text}{series=\bfseries}
\setbeamercolor{alerted text}{fg=yellow}

% Extra highlighting commands
\usepackage{color}

\newcommand{\shade}[1] {%
  {\color{medgrey} #1}}

\newcommand{\lightshade}[1] {%
  {\color{lightgrey} #1}}

\newcommand{\notealert}[1] {%
  {\color{blue} #1}}

% Set the footnote size
\setbeamerfont{footnote}{size=\tiny}

\begin{document}

\title[]{Putting Probabilities First: How Hilbert Space Generates and
  Constrains Them}
\author[M. Cuffaro]{Michael Cuffaro{\color{lightgrey}\inst{1,2,3}}}
\institute[]{%
  \color{lightgrey}
  \inst{1} Rotman Institute of Philosophy, University of Western Ontario\\[2pt]%
  \inst{2} Munich Center for Mathematical Philosophy, LMU Munich\\[2pt]%
  \inst{3} Institute for Quantum Optics and Quantum Information Vienna}
\date{\mbox{ } \\[20pt] March 11, 2019}


\begin{frame}[plain,noframenumbering]
\titlepage
\insertsubtitle{
  \begin{center}
    Politecnico di Milano, Milan, Italy\\[10pt]
    {\scriptsize\normalem \emph{\color{medgrey}{Powered by \LaTeX}}}
  \end{center}}
\end{frame}

\section*{First Section}

\subsection*{1}
\begin{frame}
\frametitle{Goals of this paper}
\begin{itemize}
\item Clarify Wittgenstein's views on probability statements.
\item Defend Wittgenstein's views against misunderstandings.
\item Point out the limitations of Wittgenstein's conception.
\end{itemize}
\end{frame}

\subsection*{2}
\begin{frame}
\frametitle{Classical interpretation of Probability (Laplace, 1825)}
\begin{itemize}
\item Principle of Insufficient Reason (a.k.a. Principle of Indifference)
  \begin{itemize}
  \item If there is no reason to prefer one outcome over another, all are
    considered equally probable.
  \end{itemize}
\item Definition: $P = \frac{\mbox{\# favourable cases}}{\mbox{\# total
    cases}}$
\item Problems
  \begin{itemize}
  \item For a biased die the outcomes are not equally probable.
  \item Problem: The probability of death for a 40 year old insurance salesman
    is 0.011. How is this defined? Which are the favourable outcomes? Which are
    the total possible cases?
  \end{itemize}
\end{itemize}
\note{GNU Emacs is the most popular and widely used of the Emacs family of text
  editors. It is also the most powerful and flexible. Unlike other text
  editors, GNU Emacs is a complete working environment\textemdash you can stay
  within Emacs all day without leaving.}
\end{frame}

\subsection*{3}
\begin{frame}
\frametitle{Frequentist Interpretation of Probability (Ellis, Venn, von Mises)}
\begin{itemize}
\item Probability is defined with respect to an observation of repeated trials,
  e.g., repeated rolls of a biased die:
  \begin{itemize}
  \item Relative frequency of 6 with a biased die after a sequence of 1000
    rolls = $\frac{320}{1000} = 0.32$
  \item After $10000$ rolls: 0.323.
  \item After $100000$ rolls: 0.3231
  \item etc.
  \item Definition of the probability of getting a 6 with this die (von Mises):
    \textbf{the limiting value of the relative frequency of 6 with respect to
      an infinite sequence of rolls with this die.}
  \end{itemize}
\end{itemize}
\end{frame}

\begin{frame}
  \frametitle{}
  \begin{columns} 
    \begin{column}{0.5\textwidth}    
      some text here some text here some text here some text here some text
      here
    \end{column} 
    \begin{column}{0.5\textwidth}       
      \begin{figure}[p]
        \caption{a}
        % \includegraphics[width=0.5\textwidth]{test} %This is my picture
      \end{figure}
    \end{column} 
  \end{columns} 
\end{frame}

%%%%%%%%%%%%%%%%%%%%%%%%%%%%%%%%%%%%%%%%%%%%%%%%%%%%%%%%%%%%%%%%%%%%%%%%%%%%%%%%%%%%%%%%%%%%%%%%%%%%
% Set the ``continuation'' text to null. Comment out if the bibliography
% requires more than one slide. It is also possible to customise this text.
%\setbeamertemplate{frametitle continuation}{}

%\begin{frame}[plain,allowframebreaks,noframenumbering]{Works Cited}
%%\frametitle{}
%\def\newblock{}
%\scriptsize
%\bibliographystyle{apa-good}
%\bibliography{Bibliography}{}
%\end{frame}

%\begin{frame}[plain,noframenumbering]
%\frametitle{}
%\vfill
%{\LARGE \begin{center} \textbf{Thanks!} \end{center}}
%\vfill
%\end{frame}

\end{document}
