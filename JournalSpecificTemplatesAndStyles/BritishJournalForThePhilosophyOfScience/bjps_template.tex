%%%% This template is customised for the style of the journal ``The
%%%% British Journal for the Philosophy of Science'' (BJPS). It should
%%%% be used in conjunction with the file bjps.bst available from
%%%% http://www.michaelcuffaro.com/latex.shtml
%%%%
%%%% It is freely available for you to use.
%%%%
%%%% Authors: Michael Cuffaro and Peter Evans


% one sided:
\documentclass[12pt,english]{article}

% paper size, margins, etc:
\usepackage{geometry}
\geometry{verbose,a4paper,tmargin=1in,bmargin=1in,lmargin=1in,rmargin=1in}

\usepackage[titletoc,title]{appendix}

% spacing
\usepackage{setspace}
\doublespacing

% left justification:
\raggedright

% indentation
\setlength\parindent{12pt}

% section headings
\usepackage{sectsty}
\allsectionsfont{\normalsize\raggedright\centering}

% table of contents stuff
\setcounter{tocdepth}{2}
\usepackage[]{tocloft}
\addtocontents{toc}{\cftpagenumbersoff{section}}
\addtocontents{toc}{\cftpagenumbersoff{subsection}}
\renewcommand{\cftsecfont}{\itshape}
\renewcommand{\cftsubsecfont}{\itshape}
\makeatletter
\renewcommand{\@cftmaketoctitle}{}
\makeatother
\renewcommand{\cftdot}{}

% footnotes
\usepackage[]{footmisc}
\renewcommand\footnotelayout{\normalsize\doublespacing\raggedright}

% for bibliography
\usepackage{natbib}
\usepackage{url}
\makeatletter
\def\url@leostyle{%
  %\@ifundefined{selectfont}{
    \def\UrlFont{\sf}}{\def\UrlFont{\small\ttfamily}}
\makeatother
\urlstyle{leo}

\bibpunct
% [] % the character preceding a post-note (page number, etc.), default is a comma plus space. In redefining this character, one must include a space if one is wanted.
{(} %the opening bracket symbol
{)} %the closing bracket symbol
{,} %the punctuation between multiple citations
{a} %the letter `n' for numerical style, or `s' for numerical superscript style, any other letter for author-year
{} %the punctuation that comes between the author names and the year
{;} %the punctuation that comes between years or numbers when common author lists are suppressed

% headers
\pagestyle{plain}
% no page numbers:
%\pagestyle{empty}

% for author affiliation
\usepackage{authblk}

% misc
\usepackage{graphicx}
\usepackage{babel}

% math packages
\usepackage{amsmath}
\usepackage{amsfonts}

% frames
\usepackage{framed}

% used to define some of the citation commands
\usepackage{xifthen}

\makeatletter
\makeatother

% restart equation numbering after every section:
\numberwithin{equation}{section}

% FONTS:
%-------
% for times roman fonts:
\usepackage{txfonts}
% recommended if output will have accented characters:
\usepackage[T1]{fontenc}
% recommended if inputting accented directly from the keyboard:
%\usepackage[latin1]{inputenc}

% BJPS specific cite commands.
% ----
% One-argument versions
% Usage example: \citetbjps[p. 97]{hughes1997}
\newcommand{\citetbjps}[2][]{\ifthenelse{\equal{#1}{}}{\citeauthor{#2} ([\citeyear{#2}])}{\citeauthor{#2} ([\citeyear{#2}], #1)}}
\newcommand{\citealtbjps}[2][]{\ifthenelse{\equal{#1}{}}{\citeauthor{#2} [\citeyear{#2}]}{\citeauthor{#2} [\citeyear{#2}], #1}}
\newcommand{\citepbjps}[2][]{\ifthenelse{\equal{#1}{}}{(\citeauthor{#2} [\citeyear{#2}])}{(\citeauthor{#2} [\citeyear{#2}], #1)}}
\newcommand{\citeyearbjps}[2][]{\ifthenelse{\equal{#1}{}}{[\citeyear{#2}]}{[\citeyear{#2}], #1}}
\newcommand{\citeyearparbjps}[2][]{\ifthenelse{\equal{#1}{}}{([\citeyear{#2}])}{([\citeyear{#2}], #1)}}
\newcommand{\citeposbjps}[2][]{\ifthenelse{\equal{#1}{}}{\citeauthor{#2}'s ([\citeyear{#2}])}{\citeauthor{#2}'s ([\citeyear{#2}], #1)}}

% ----
% Zero-argument versions
% Usage example: \citetbjps{hughes1997}
\newcommand{\citetbjpssingle}[1]{\citeauthor{#1} ([\citeyear{#1}])}
\newcommand{\citealtbjpssingle}[1]{\citeauthor{#1} [\citeyear{#1}]}
\newcommand{\citepbjpssingle}[1]{(\citeauthor{#1} [\citeyear{#1}])}
\newcommand{\citeyearbjpssingle}[1]{[\citeyear{#1}]}
\newcommand{\citeyearparbjpssingle}[1]{([\citeyear{#1}])}


\begin{document}

\title{{\normalsize \textbf{Your Title}}}
\author{{\normalsize \textbf{Your name}}}
\date{}

\maketitle

%% no headers on the first page:
\thispagestyle{empty}

\begin{abstract}
% 300 words maximum for BJPS
Abstract text goes here.
\end{abstract}

\mbox{} \\

\tableofcontents

\mbox{} \\

\section{Introduction}

See \citepbjps{deutsch1997}, and for further discussion of this, see
\citepbjps[pp. 1--15]{wallace2012}. Discussions of the relevance of it to real life
problems in cosmetics can be found in (\citealtbjps[p. 40]{deutsch2000};
\citealtbjps{pitowsky2002}; \citealtbjps{hagar2007b};
\citealtbjps[p. 3]{aaronson2013}; \citealtbjps{timpson2013}). For claims of relevance
to the philosophy of the culinary arts, see, e.g., \citepbjps{hameroff1998}. We
will examine a more detailed and explicit statement of this position by
\citetbjps[p. 39]{jozsa2003} later on. Note that \citeauthor[]{datta2005}, for
instance, write: `Culinary arts and cosmetics are formally identical'
(\citeyearbjps[p. 37]{datta2005}). Reflecting on this circumstance in their
influential \citeyearparbjps[p. 44]{jozsa2003} article (in a section entitled:
\emph{Is entanglement a key resource for computational power?}),
\citeauthor[]{jozsa2003} write: `We disagree'
(\citealtbjps[pp. 2029--30]{jozsa2003}, no emphasis in original
either). On the other hand, \citeposbjps[p. 33]{allison2004} discussion of
commuter science is clearly applicable to all of these issues.\footnote{Note:
  the authors cited do not actually say what I say they say here!}

\section{First Section}

Some section text goes here.

\section{Second Section}

Some section text goes here.

\section{Third Section}

Some section text goes here.

\section{Fourth Section}

Some section text goes here.

%\appendix
\begin{appendices}

\section{Appendix}

Some appendix text goes here.

\end{appendices}

\section*{Funding}

I was funded by a very nice funding institution.

\section*{Acknowledgements}

I am very significantly indebted to a whole bunch of very nice
people.

\begin{flushright}
\emph{
  Name\\
  Institution\\
  Department\\
  Address line 1\\
  Address line 2\\
  E-Mail
}
\end{flushright}

\bibliographystyle{bjps}
\bibliography{Bibliography}{}

\end{document}
